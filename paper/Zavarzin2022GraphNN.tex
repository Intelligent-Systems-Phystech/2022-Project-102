\documentclass[12pt, twoside]{article}
\usepackage{jmlda}
\newcommand{\hdir}{.}

\begin{document}

\title
    {Регрессия на паре графов}
\author
    {Артём~Заварзин, Филипп Никитин, Вадим Стрижов, Александр Исаев} % основной список авторов, выводимый в оглавление
\abstract
    {
    Рассматривается задача регрессии на паре графов. В паре каждой вершине одного графа соответствует вершина второго графа. Требуется установить оптимальную архитектуру графовой нейронной сети, учитывающую данный порядок, заданный на вершинах.
    
    В базовом алгоритме строится отдельный вектор-эмбеддинг для каждого графа, а затем данные вектора конкатенируются. В данном случае явно не используется информация о соответствии вершин в графах. На примере архитектуры графовой нейронной сети с фиксированными гиперпараметрами с теоретической и практической точки зрения будет рассмотрен способ добавления в графовую нейронную сеть информации об отношении графов.
	
\bigskip
}

\maketitle
\linenumbers

\section{Введение}

\end{document}
